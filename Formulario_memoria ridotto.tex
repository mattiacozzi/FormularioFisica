\documentclass[a4paper,11pt,italian]{article}
\usepackage[T1]{fontenc}
\usepackage[utf8]{inputenc}
\usepackage[italian]{babel}
\usepackage{amsmath}
\usepackage{mathrsfs}
\usepackage{hyperref}
%%%--- impostazioni font
\usepackage{sansmathfonts}
\usepackage[scaled=0.85]{helvet}
\renewcommand{\rmdefault}{\sfdefault}
\usepackage[font=footnotesize,width=.75\textwidth]{caption}

%%%--- frazioni carine \sfrac{}{}
\usepackage{xfrac}

%%%--- interlinea
\usepackage{setspace}
\setstretch{1.3}

%%%--- multicolonna
\usepackage{multicol}

%%%--- figure in orizzontale
\usepackage{rotating}

%%%--- margini e bordo
\usepackage[margin=2.7cm]{geometry}
% \usepackage{showframe}

%%%--- impostazioni tikz e grafici
\usepackage{pgf,tikz,pgfplots,bm,pgf-spectra,graphicx,timelines}
\usepackage{circuitikz}
\usetikzlibrary{angles,quotes,arrows,shapes,decorations.markings}
\tikzset{fleche/.style args={#1:#2}{postaction=decorate,decoration={name=markings,mark=at position #1 with {\arrow[#2,scale=2]{>}}},},}
\pgfplotsset{compat=1.15}
\usepgfplotslibrary{units,fillbetween}

%%%--- info documento
\title{\textbf{\Huge \color{colorSitoScuro}Formulario di Fisica}}
\author{\emph{\LARGE \color{colorSitoScuro}Mattia Cozzi}}
\date{}

%%%--- impostazioni dei colori
\usepackage{xcolor}
\usepackage{sectsty}
\usepackage{enumitem}
%%%--- TEAL
\definecolor{colorSito}{rgb}{.04,.212,.22}
\definecolor{colorSitoChiaro}{rgb}{.20,.408,.416}
\definecolor{colorSitoScuro}{rgb}{.0,.133,.141}
%%%--- BORDEAUX
% \definecolor{colorSito}{rgb}{.612,.067,.192}
% \definecolor{colorSitoChiaro}{rgb}{.69,.145,.271}
% \definecolor{colorSitoScuro}{rgb}{.22,.0,.0}
%
\sectionfont{\color{colorSito}}  % sets colour of sections
\subsectionfont{\color{colorSitoChiaro}}  % sets colour of subsections
\subsubsectionfont{\color{colorSitoChiaro}}  % sets colour of subsubsections
\setlist[description]{format=\textcolor{colorSitoScuro}} % sets colour of definition lists items

\begin{document}
\begin{enumerate}
\item Funzioni goniometriche
\item Scomposizione di un vettore
\item Forma cartesiana di un vettore
\item Modulo di un vettore, note le componenti
\item Direzione di un vettore 
\item Somma tra vettori, forma cartesiana
\item Differenza tra vettori, forma cartesiana
\item Prodotto di uno scalare per un vettore, forma cartesiana
\item Prodotto scalare, forma polare
\item Prodotto vettoriale, forma polare
\item Velocità della luce nel vuoto
\item Costante dielettrica del vuoto
\item Costante di Coulomb
\item Permeabilità magnetica del vuoto
\item Carica elementare
\item Costante di Planck
\item Densità di un corpo
\item Velocità media
\item Velocità istantanea
\item Conversione tra km/h e m/s
\item Accelerazione media
\item Accelerazione istantanea
\item Leggi del moto rettilineo uniforme
\item Leggi del moto accelerato
\item Frequenza del moto circolare
\item Velocità angolare (pulsazione) del moto circolare
\item Velocità tangenziale
\item Accelerazione centripeta
\item Forza centripeta
\item Legge del moto armonico 
\item Secondo principio della dinamica (legge fondamentale della dinamica)
\item Condizione di equilibrio per corpi puntiformi
\item Forza peso
\item Attrito statico
\item Forza di richiamo di una molla (legge di Hooke)
\item Definizione di lavoro
\item Il lavoro è una variazione di energia
\item Potenza media
\item Energia cinetica di traslazione
\item Energia potenziale gravitazionale
\item Quantità di moto
\item Legge di gravitazione universale
\item Accelerazione di gravità sulla superficie della Terra 
\item Pressione 
\item Conversione Celsius-Kelvin
\item Legge fondamentale della calorimetria
\item Primo principio della termodinamica
\item Frequenza 
\item Pulsazione dell'onda
\item Velocità di propagazione dell'onda
\item Legge oraria delle onde in un punto fissato 
\item Indice di rifrazione di un mezzo materiale
\item Legge della rifrazione (legge di Snell)
\item Riflessione totale, angolo limite
\item Legge di Coulomb
\item Costante elettrica del vuoto 
\item Costante dielettrica del vuoto
\item Definizione di campo elettrico
\item Campo elettrico generato da una carica puntiforme
\item Forza subita da una carica in un campo elettrico
\item Flusso del campo elettrico
\item Teorema di Gauss per il campo elettrico
\item Energia potenziale elettrica di un sistema di due cariche
\item Potenziale elettrico
\item Energia acquistata/perduta da una carica sottoposta a tensione
\item Differenza di potenziale (tensione) tra i punti \textit{A} e \textit{B}
\item Potenziale elettrico generato da una carica \textit{Q} a distanza \textit{r}
\item Circuitazione del campo elettrico
\item Capacità di un condensatore
\item Capacità totale per condensatori in parallelo
\item Capacità totale per condensatori in serie
\item Intensità media di corrente
\item Intensità istantanea di corrente
\item Prima legge di Ohm
\item Resistenza totale per resistori in parallelo
\item Resistenza totale per resistori in serie
\item Potenza dissipata da una resistenza
\item Energia dissipata per effetto Joule
\item Elettronvolt
\item Legge di Ampère
\item Permeabilità magnetica del vuoto
\item Forza subita da un filo in un campo magnetico
\item Legge di Biot-Savart
\item Forza di Lorentz
\item Flusso del campo magnetico
\item Teorema di Gauss per il campo magnetico
\item Circuitazione del campo magnetico
\item Teorema di Ampère
\item Legge di Faraday-Neumann-Lenz, fem indotta media
\item Fem indotta istantanea
\item Induttanza
\item Autoinduzione
\item Forza elettromotrice in corrente alternata
\item Corrente in regime alternato
\item Valori efficaci in corrente alternata
\item Potenza media prodotta in corrente alternata
\item Trasformatori
\item Circuitazione del campo elettrico indotto (FNL)
\item Corrente di spostamento
\item Equazioni nel caso statico
\item Velocità di un'onda elettromagnetica nel vuoto
\item Frequenza e lunghezza d'onda 
\item Ampiezze di \textit{E} e di \textit{B}
\item Coefficiente di dilatazione (fattore di Lorentz)
\item Dilatazione dei tempi
\item Contrazione delle lunghezze parallele al moto
\item Composizione relativistica delle velocità
\item Effetto Doppler relativistico
\item Equivalenza massa-energia
\item Energia di quiete
\item Massa relativistica
\item Energia totale di una particella relativistica (relazione di Einstein)
\item Energia cinetica relativistica
\item Quantità di moto relativistica
\item Quantità di moto della luce
\item Costante di Planck
\item Energia di un fotone
\item Relazione di De Broglie
\item Costante di Planck ridotta
\item Principio di indeterminazione di Heisenberg  
\end{enumerate}


GIALLI OK
\begin{enumerate}
\item Conversione tra km/h e m/s
\item Forma cartesiana di un vettore
\item Funzioni goniometriche
\item Quantità di moto
\item Prodotto vettoriale, forma polare
\item Capacità totale per condensatori in serie
\item Fem indotta istantanea
\item Primo principio della termodinamica
\item Prodotto scalare, forma polare
\item Permeabilità magnetica del vuoto
\item Accelerazione di gravità sulla superficie della Terra 
\item Forza peso
\item Definizione di campo elettrico
\item Forza subita da una carica in un campo elettrico
\item Pressione 
\item Costante dielettrica del vuoto
\item Energia potenziale elettrica di un sistema di due cariche
\item Differenza tra vettori, forma cartesiana
\item Forza di richiamo di una molla (legge di Hooke)
\item Potenza media
\item Accelerazione centripeta
\item Condizione di equilibrio per corpi puntiformi
\item Direzione di un vettore 
\item Forza di Lorentz
\item Somma tra vettori, forma cartesiana
\item Scomposizione di un vettore
\item Costante elettrica del vuoto 
\item Velocità media
\item Prima legge di Ohm
\item Accelerazione media
\item Legge di Biot-Savart
\item Teorema di Gauss per il campo magnetico
\item Legge di Faraday-Neumann-Lenz, fem indotta media
\item Resistenza totale per resistori in parallelo
\item Velocità della luce nel vuoto
\item Forza subita da un filo in un campo magnetico
\item Attrito statico
\item Legge fondamentale della calorimetria
\item Legge di Coulomb
\item Circuitazione del campo magnetico
\item Potenziale elettrico
\item Resistenza totale per resistori in serie
\item Velocità di propagazione di un'onda
\item Definizione di lavoro
\item Intensità istantanea di corrente
\end{enumerate}



ROSSI OK
\begin{enumerate}
\item Intensità media di corrente
\item Prima legge di Ohm
\item Teorema di Gauss per il campo elettrico
\item Definizione di lavoro
\item Condizione di equilibrio per corpi puntiformi
\item Forza subita da una carica in un campo elettrico
\item Forza di Lorentz
\item Accelerazione media
\item Teorema di Gauss per il campo magnetico
\item Velocità tangenziale
\item Legge fondamentale della calorimetria
\item Velocità angolare (pulsazione) del moto circolare
\item Teorema di Ampère
\item Forza peso
\item Accelerazione centripeta
\item Forza subita da un filo in un campo magnetico
\item Energia cinetica di traslazione
\item Leggi del moto accelerato
\item Campo elettrico generato da una carica puntiforme
\item Legge di Faraday-Neumann-Lenz, fem indotta media
\item Energia acquistata/perduta da una carica sottoposta a tensione
\item Fem indotta istantanea
\item Definizione di campo elettrico
\item Velocità media
\item Accelerazione istantanea
\item Legge di Ampère
\item Resistenza totale per resistori in parallelo
\item Somma tra vettori, forma cartesiana
\item Potenza dissipata da una resistenza
\item Funzioni goniometriche
\item Prodotto scalare, forma polare
\item Intensità istantanea di corrente
\item Circuitazione del campo magnetico
\item Costante dielettrica del vuoto
\item Frequenza 
\item Direzione di un vettore 
\item Resistenza totale per resistori in serie
\item Secondo principio della dinamica (legge fondamentale della dinamica)
\item Quantità di moto
\item Prodotto di uno scalare per un vettore, forma cartesiana
\item Velocità della luce nel vuoto
\item Potenza media
\item Velocità istantanea
\item Pressione
\item Costante elettrica del vuoto
\end{enumerate}




VERDI OK
\begin{enumerate}
\item Differenza tra vettori, forma cartesiana
\item Leggi del moto rettilineo uniforme
\item Forza subita da un filo in un campo magnetico
\item Leggi del moto accelerato
\item Attrito statico
\item Capacità totale per condensatori in serie
\item Forza di richiamo di una molla (legge di Hooke)
\item Potenza dissipata da una resistenza
\item Costante elettrica del vuoto 
\item Forma cartesiana di un vettore
\item Somma tra vettori, forma cartesiana
\item Velocità tangenziale
\item Frequenza 
\item Teorema di Ampère
\item Flusso del campo elettrico
\item Circuitazione del campo magnetico
\item Prodotto vettoriale, forma polare
\item Accelerazione centripeta
\item Prima legge di Ohm
\item Secondo principio della dinamica (legge fondamentale della dinamica)
\item Forza subita da una carica in un campo elettrico
\item Accelerazione istantanea
\item Velocità di propagazione di un'onda
\item Teorema di Gauss per il campo elettrico
\item Costante dielettrica del vuoto
\item Capacità totale per condensatori in parallelo
\item Potenziale elettrico
\item Resistenza totale per resistori in serie
\item Scomposizione di un vettore
\item Conversione tra km/h e m/s
\item Forza di Lorentz
\item Velocità istantanea
\item Il lavoro è una variazione di energia
\item Accelerazione media
\item Energia potenziale gravitazionale
\item Modulo di un vettore, note le componenti
\item Legge di Faraday-Neumann-Lenz, fem indotta media
\item Primo principio della termodinamica
\item Definizione di lavoro
\item Velocità media
\item Velocità della luce nel vuoto
\item Pressione 
\item Definizione di campo elettrico
\item Legge di Biot-Savart
\item Densità di un corpo
\end{enumerate}





BLU
\begin{enumerate}
\item Differenza tra vettori, forma cartesiana
\item Teorema di Gauss per il campo magnetico
\item Prima legge di Ohm
\item Costante elettrica del vuoto 
\item Flusso del campo magnetico
\item Teorema di Ampère 
\item Circuitazione del campo magnetico
\item Fem indotta istantanea
\item Attrito statico
\item Prodotto di uno scalare per un vettore, forma cartesiana
\item Resistenza totale per resistori in serie
\item Scomposizione di un vettore
\item Energia acquistata/perduta da una carica sottoposta a tensione
\item Forza di Lorentz
\item Permeabilità magnetica del vuoto
\item Conversione tra km/h e m/s
\item Flusso del campo elettrico
\item Legge fondamentale della calorimetria
\item Accelerazione centripeta
\item Velocità della luce nel vuoto
\item Leggi del moto rettilineo uniforme
\item Accelerazione istantanea
\item Velocità media
\item Accelerazione di gravità sulla superficie della Terra 
\item Capacità totale per condensatori in serie
\item Il lavoro è una variazione di energia
\item Intensità istantanea di corrente
\item Legge di Faraday-Neumann-Lenz, fem indotta media
\item Modulo di un vettore, note le componenti
\item Legge di gravitazione universale
\item Forza di richiamo di una molla (legge di Hooke)
\item Frequenza
\item Direzione di un vettore 
\item Circuitazione del campo elettrico
\item Velocità angolare (pulsazione) del moto circolare
\item Somma tra vettori, forma cartesiana
\item Pulsazione dell'onda
\item Pressione 
\item Intensità media di corrente
\item Forza subita da un filo in un campo magnetico
\item Velocità tangenziale
\item Velocità istantanea
\item Energia cinetica di traslazione
\item Forza peso
\item Prodotto scalare, forma polare
\end{enumerate}
\end{document}

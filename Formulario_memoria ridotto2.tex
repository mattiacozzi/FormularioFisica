\documentclass[a4paper,11pt,italian]{article}
\usepackage[T1]{fontenc}
\usepackage[utf8]{inputenc}
\usepackage[italian]{babel}
\usepackage{amsmath}
\usepackage{mathrsfs}
\usepackage{hyperref}
%%%--- impostazioni font
\usepackage{sansmathfonts}
\usepackage[scaled=0.85]{helvet}
\renewcommand{\rmdefault}{\sfdefault}
\usepackage[font=footnotesize,width=.75\textwidth]{caption}

%%%--- frazioni carine \sfrac{}{}
\usepackage{xfrac}

%%%--- interlinea
\usepackage{setspace}
\setstretch{1.3}

%%%--- multicolonna
\usepackage{multicol}

%%%--- figure in orizzontale
\usepackage{rotating}

%%%--- margini e bordo
\usepackage[margin=2.7cm]{geometry}
% \usepackage{showframe}

%%%--- impostazioni tikz e grafici
\usepackage{pgf,tikz,pgfplots,bm,pgf-spectra,graphicx,timelines}
\usepackage{circuitikz}
\usetikzlibrary{angles,quotes,arrows,shapes,decorations.markings}
\tikzset{fleche/.style args={#1:#2}{postaction=decorate,decoration={name=markings,mark=at position #1 with {\arrow[#2,scale=2]{>}}},},}
\pgfplotsset{compat=1.15}
\usepgfplotslibrary{units,fillbetween}

%%%--- info documento
\title{\textbf{\Huge \color{colorSitoScuro}Formulario di Fisica}}
\author{\emph{\LARGE \color{colorSitoScuro}Mattia Cozzi}}
\date{}

%%%--- impostazioni dei colori
\usepackage{xcolor}
\usepackage{sectsty}
\usepackage{enumitem}
%%%--- TEAL
\definecolor{colorSito}{rgb}{.04,.212,.22}
\definecolor{colorSitoChiaro}{rgb}{.20,.408,.416}
\definecolor{colorSitoScuro}{rgb}{.0,.133,.141}
%%%--- BORDEAUX
% \definecolor{colorSito}{rgb}{.612,.067,.192}
% \definecolor{colorSitoChiaro}{rgb}{.69,.145,.271}
% \definecolor{colorSitoScuro}{rgb}{.22,.0,.0}
%
\sectionfont{\color{colorSito}}  % sets colour of sections
\subsectionfont{\color{colorSitoChiaro}}  % sets colour of subsections
\subsubsectionfont{\color{colorSitoChiaro}}  % sets colour of subsubsections
\setlist[description]{format=\textcolor{colorSitoScuro}} % sets colour of definition lists items

\begin{document}



GIALLI OK
\begin{enumerate}
{Conversione tra km/h e m/s}
{Forma cartesiana di un vettore}
{Funzioni goniometriche}
{Quantità di moto}
{Prodotto vettoriale, forma polare}
{Capacità totale per condensatori in serie}
{Fem indotta istantanea}
{Primo principio della termodinamica}
{Prodotto scalare, forma polare}
{Permeabilità magnetica del vuoto}
{Accelerazione di gravità sulla superficie della Terra }
{Forza peso}
{Definizione di campo elettrico}
{Forza subita da una carica in un campo elettrico}
{Pressione }
{Costante dielettrica del vuoto}
{Energia potenziale elettrica di un sistema di due cariche}
{Differenza tra vettori, forma cartesiana}
{Forza di richiamo di una molla (legge di Hooke)}
{Potenza media}
{Accelerazione centripeta}
{Condizione di equilibrio per corpi puntiformi}
{Direzione di un vettore }
{Forza di Lorentz}
{Somma tra vettori, forma cartesiana}
{Scomposizione di un vettore}
{Costante elettrica del vuoto }
{Velocità media}
{Prima legge di Ohm}
{Accelerazione media}
{Legge di Biot-Savart}
{Teorema di Gauss per il campo magnetico}
{Legge di Faraday-Neumann-Lenz, fem indotta media}
{Resistenza totale per resistori in parallelo}
{Velocità della luce nel vuoto}
{Forza subita da un filo in un campo magnetico}
{Attrito statico}
{Legge fondamentale della calorimetria}
{Legge di Coulomb}
{Circuitazione del campo magnetico}
{Potenziale elettrico}
{Resistenza totale per resistori in serie}
{Velocità di propagazione di un'onda}
{Definizione di lavoro}
{Intensità istantanea di corrente}
\end{enumerate}



ROSSI OK
\begin{enumerate}
{Intensità media di corrente}
{Prima legge di Ohm}
{Teorema di Gauss per il campo elettrico}
{Definizione di lavoro}
{Condizione di equilibrio per corpi puntiformi}
{Forza subita da una carica in un campo elettrico}
{Forza di Lorentz}
{Accelerazione media}
{Teorema di Gauss per il campo magnetico}
{Velocità tangenziale}
{Legge fondamentale della calorimetria}
{Velocità angolare (pulsazione) del moto circolare}
{Teorema di Ampère}
{Forza peso}
{Accelerazione centripeta}
{Forza subita da un filo in un campo magnetico}
{Energia cinetica di traslazione}
{Leggi del moto accelerato}
{Campo elettrico generato da una carica puntiforme}
{Legge di Faraday-Neumann-Lenz, fem indotta media}
{Energia acquistata/perduta da una carica sottoposta a tensione}
{Fem indotta istantanea}
{Definizione di campo elettrico}
{Velocità media}
{Accelerazione istantanea}
{Legge di Ampère}
{Resistenza totale per resistori in parallelo}
{Somma tra vettori, forma cartesiana}
{Potenza dissipata da una resistenza}
{Funzioni goniometriche}
{Prodotto scalare, forma polare}
{Intensità istantanea di corrente}
{Circuitazione del campo magnetico}
{Costante dielettrica del vuoto}
{Frequenza }
{Direzione di un vettore }
{Resistenza totale per resistori in serie}
{Secondo principio della dinamica (legge fondamentale della dinamica)}
{Quantità di moto}
{Prodotto di uno scalare per un vettore, forma cartesiana}
{Velocità della luce nel vuoto}
{Potenza media}
{Velocità istantanea}
{Pressione}
{Costante elettrica del vuoto}
\end{enumerate}




VERDI OK
\begin{enumerate}
{Differenza tra vettori, forma cartesiana}
{Leggi del moto rettilineo uniforme}
{Forza subita da un filo in un campo magnetico}
{Leggi del moto accelerato}
{Attrito statico}
{Capacità totale per condensatori in serie}
{Forza di richiamo di una molla (legge di Hooke)}
{Potenza dissipata da una resistenza}
{Costante elettrica del vuoto }
{Forma cartesiana di un vettore}
{Somma tra vettori, forma cartesiana}
{Velocità tangenziale}
{Frequenza }
{Teorema di Ampère}
{Flusso del campo elettrico}
{Circuitazione del campo magnetico}
{Prodotto vettoriale, forma polare}
{Accelerazione centripeta}
{Prima legge di Ohm}
{Secondo principio della dinamica (legge fondamentale della dinamica)}
{Forza subita da una carica in un campo elettrico}
{Accelerazione istantanea}
{Velocità di propagazione di un'onda}
{Teorema di Gauss per il campo elettrico}
{Costante dielettrica del vuoto}
{Capacità totale per condensatori in parallelo}
{Potenziale elettrico}
{Resistenza totale per resistori in serie}
{Scomposizione di un vettore}
{Conversione tra km/h e m/s}
{Forza di Lorentz}
{Velocità istantanea}
{Il lavoro è una variazione di energia}
{Accelerazione media}
{Energia potenziale gravitazionale}
{Modulo di un vettore, note le componenti}
{Legge di Faraday-Neumann-Lenz, fem indotta media}
{Primo principio della termodinamica}
{Definizione di lavoro}
{Velocità media}
{Velocità della luce nel vuoto}
{Pressione }
{Definizione di campo elettrico}
{Legge di Biot-Savart}
{Densità di un corpo}
\end{enumerate}





BLU OK
\begin{enumerate}
{Differenza tra vettori, forma cartesiana}
{Teorema di Gauss per il campo magnetico}
{Prima legge di Ohm}
{Costante elettrica del vuoto }
{Flusso del campo magnetico}
{Teorema di Ampère }
{Circuitazione del campo magnetico}
{Fem indotta istantanea}
{Attrito statico}
{Prodotto di uno scalare per un vettore, forma cartesiana}
{Resistenza totale per resistori in serie}
{Scomposizione di un vettore}
{Energia acquistata/perduta da una carica sottoposta a tensione}
{Forza di Lorentz}
{Permeabilità magnetica del vuoto}
{Conversione tra km/h e m/s}
{Flusso del campo elettrico}
{Legge fondamentale della calorimetria}
{Accelerazione centripeta}
{Velocità della luce nel vuoto}
{Leggi del moto rettilineo uniforme}
{Accelerazione istantanea}
{Velocità media}
{Accelerazione di gravità sulla superficie della Terra }
{Capacità totale per condensatori in serie}
{Il lavoro è una variazione di energia}
{Intensità istantanea di corrente}
{Legge di Faraday-Neumann-Lenz, fem indotta media}
{Modulo di un vettore, note le componenti}
{Legge di gravitazione universale}
{Forza di richiamo di una molla (legge di Hooke)}
{Frequenza}
{Direzione di un vettore }
{Circuitazione del campo elettrico}
{Velocità angolare (pulsazione) del moto circolare}
{Somma tra vettori, forma cartesiana}
{Pulsazione dell'onda}
{Pressione }
{Intensità media di corrente}
{Forza subita da un filo in un campo magnetico}
{Velocità tangenziale}
{Velocità istantanea}
{Energia cinetica di traslazione}
{Forza peso}
{Prodotto scalare, forma polare}
\end{enumerate}
\end{document}

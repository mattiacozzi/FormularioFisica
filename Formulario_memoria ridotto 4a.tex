\documentclass[a4paper,11pt,italian]{article}
\usepackage[T1]{fontenc}
\usepackage[utf8]{inputenc}
\usepackage[italian]{babel}
\usepackage{amsmath}
\usepackage{mathrsfs}
\usepackage{hyperref}
%%%--- impostazioni font
\usepackage{sansmathfonts}
\usepackage[scaled=0.85]{helvet}
\renewcommand{\rmdefault}{\sfdefault}
\usepackage[font=footnotesize,width=.75\textwidth]{caption}

%%%--- frazioni carine \sfrac{}{}
\usepackage{xfrac}

%%%--- interlinea
\usepackage{setspace}
\setstretch{1.3}

%%%--- multicolonna
\usepackage{multicol}

%%%--- figure in orizzontale
\usepackage{rotating}

%%%--- margini e bordo
\usepackage[margin=2.7cm]{geometry}
% \usepackage{showframe}

%%%--- impostazioni tikz e grafici
\usepackage{pgf,tikz,pgfplots,bm,pgf-spectra,graphicx,timelines}
\usepackage{circuitikz}
\usetikzlibrary{angles,quotes,arrows,shapes,decorations.markings}
\tikzset{fleche/.style args={#1:#2}{postaction=decorate,decoration={name=markings,mark=at position #1 with {\arrow[#2,scale=1.5]{latex}}},},}
\pgfplotsset{compat=1.15}
\usepgfplotslibrary{units,fillbetween}

%%%--- info documento
\title{\textbf{\Huge \color{colorSitoScuro}Formulario di Fisica}}
\author{\LARGE \color{colorSitoScuro}Mattia Cozzi}
\date{\large \color{colorSitoChiaro}\href{http://mattiacozzi.altervista.org/}{mattiacozzi.altervista.org}}

%%%--- riferimenti intelligenti
\usepackage[noabbrev]{cleveref} %opzione capitalize per tutto maiuscolo

%%%---righelli migliori per tabelle
\usepackage{booktabs}

%%%--- impostazioni dei colori
\usepackage{xcolor}
\usepackage{sectsty}
\usepackage{enumitem}
%%%--- TEAL
\definecolor{colorSito}{rgb}{.04,.212,.22}
\definecolor{colorSitoChiaro}{rgb}{.20,.408,.416}
\definecolor{colorSitoScuro}{rgb}{.0,.133,.141}
%%%--- BORDEAUX
% \definecolor{colorSito}{rgb}{.612,.067,.192}
% \definecolor{colorSitoChiaro}{rgb}{.69,.145,.271}
% \definecolor{colorSitoScuro}{rgb}{.22,.0,.0}

\sectionfont{\color{colorSito}}  %imposta il colore delle sezioni
\subsectionfont{\color{colorSitoChiaro}}  %imposta il colore delle sottosezioni
\subsubsectionfont{\color{colorSitoChiaro}}  %imposta il colore delle sottosottosezioni
\setlist[description]{format=\textcolor{colorSitoScuro}} %imposta il colore degli elementi di una lista di definizione

%%%--- nuovi comandi

\begin{document}
\begin{enumerate}
\item Funzioni goniometriche
\item Scomposizione di un vettore
\item Forma cartesiana di un vettore
\item Modulo di un vettore, note le componenti
\item Direzione di un vettore 
\item Somma tra vettori, forma cartesiana
\item Differenza tra vettori, forma cartesiana
\item Prodotto di uno scalare per un vettore, forma cartesiana
\item Prodotto scalare, forma polare
\item Prodotto vettoriale, forma polare
\item Velocità della luce nel vuoto
\item Costante dielettrica del vuoto
\item Costante di Coulomb
\item Carica elementare
\item Densità di un corpo
\item Velocità media
\item Conversione tra km/h e m/s
\item Accelerazione media
\item Leggi del moto rettilineo uniforme
\item Leggi del moto accelerato
\item Frequenza del moto circolare
\item Velocità angolare (pulsazione) del moto circolare
\item Velocità tangenziale
\item Accelerazione centripeta
\item Forza centripeta
\item Legge del moto armonico 
\item Secondo principio della dinamica (legge fondamentale della dinamica)
\item Condizione di equilibrio per corpi puntiformi
\item Forza peso
\item Attrito statico
\item Forza di richiamo di una molla (legge di Hooke)
\item Definizione di lavoro
\item Il lavoro è una variazione di energia
\item Potenza media
\item Energia cinetica di traslazione
\item Energia potenziale gravitazionale
\item Quantità di moto
\item Legge di gravitazione universale
\item Accelerazione di gravità sulla superficie della Terra 
\item Pressione 
\item Conversione Celsius-Kelvin
\item Legge fondamentale della calorimetria
\item Primo principio della termodinamica
\item Frequenza 
\item Pulsazione dell'onda
\item Velocità di propagazione dell'onda
\item Legge oraria delle onde in un punto fissato 
\item Indice di rifrazione di un mezzo materiale
\item Legge della rifrazione (legge di Snell)
\item Riflessione totale, angolo limite
\item Legge di Coulomb
\item Costante elettrica del vuoto 
\item Definizione di campo elettrico
\item Campo elettrico generato da una carica puntiforme
\item Forza subita da una carica in un campo elettrico
\end{enumerate}

% sono 55, portare a 45

GIALLI
\begin{enumerate}
{Legge fondamentale della calorimetria}
{Energia cinetica di traslazione}
{Costante dielettrica del vuoto}
{Costante di Coulomb}
{Funzioni goniometriche}
{Velocità tangenziale}
{Legge di Coulomb}
{Condizione di equilibrio per corpi puntiformi}
{Pulsazione dell'onda}
{Velocità media}
{Conversione Celsius-Kelvin}
{Prodotto vettoriale, forma polare}
{Leggi del moto accelerato}
{Forma cartesiana di un vettore}
{Velocità della luce nel vuoto}
{Densità di un corpo}
{Legge di gravitazione universale}
{Potenza media}
{Campo elettrico generato da una carica puntiforme}
{Somma tra vettori, forma cartesiana}
{Prodotto di uno scalare per un vettore, forma cartesiana}
{Scomposizione di un vettore}
{Direzione di un vettore}
{Legge della rifrazione (legge di Snell)}
{Indice di rifrazione di un mezzo materiale}
{Secondo principio della dinamica (legge fondamentale della dinamica)}
{Modulo di un vettore, note le componenti}
{Attrito statico}
{Forza centripeta}
{Primo principio della termodinamica}
{Energia potenziale gravitazionale}
{Frequenza}
{Costante elettrica del vuoto}
{Forza subita da una carica in un campo elettrico}
{Leggi del moto rettilineo uniforme}
{Definizione di lavoro}
{Legge oraria delle onde in un punto fissato}
{Velocità angolare (pulsazione) del moto circolare}
{Forza di richiamo di una molla (legge di Hooke)}
{Quantità di moto}
{Accelerazione di gravità sulla superficie della Terra}
{Carica elementare}
{Velocità di propagazione dell'onda}
{Prodotto scalare, forma polare}
{Accelerazione media}
\end{enumerate}



ROSSI
\begin{enumerate}
{Legge fondamentale della calorimetria}
{Costante di Coulomb}
{Campo elettrico generato da una carica puntiforme}
{Velocità tangenziale}
{Direzione di un vettore}
{Frequenza}
{Quantità di moto}
{Leggi del moto rettilineo uniforme}
{Pressione}
{Accelerazione centripeta}
{Indice di rifrazione di un mezzo materiale}
{Prodotto vettoriale, forma polare}
{Velocità media}
{Primo principio della termodinamica}
{Carica elementare}
{Frequenza del moto circolare}
{Prodotto scalare, forma polare}
{Costante elettrica del vuoto}
{Energia cinetica di traslazione}
{Definizione di campo elettrico}
{Forza peso}
{Energia potenziale gravitazionale}
{Legge oraria delle onde in un punto fissato}
{Il lavoro è una variazione di energia}
{Costante dielettrica del vuoto}
{Prodotto di uno scalare per un vettore, forma cartesiana}
{Somma tra vettori, forma cartesiana}
{Potenza media}
{Legge del moto armonico}
{Forza subita da una carica in un campo elettrico}
{Condizione di equilibrio per corpi puntiformi}
{Velocità della luce nel vuoto}
{Legge di Coulomb}
{Pulsazione dell'onda}
{Modulo di un vettore, note le componenti}
{Riflessione totale, angolo limite}
{Scomposizione di un vettore}
{Secondo principio della dinamica (legge fondamentale della dinamica)}
{Conversione tra km/h e m/s}
{Leggi del moto accelerato}
{Attrito statico}
{Velocità di propagazione dell'onda}
{Legge della rifrazione (legge di Snell)}
{Accelerazione di gravità sulla superficie della Terra}
{Legge di gravitazione universale}
\end{enumerate}




VERDI
\begin{enumerate}
{Velocità della luce nel vuoto}
{Legge fondamentale della calorimetria}
{Frequenza}
{Legge di Coulomb}
{Legge di gravitazione universale}
{Forza peso}
{Potenza media}
{Legge oraria delle onde in un punto fissato}
{Secondo principio della dinamica (legge fondamentale della dinamica)}
{Definizione di lavoro}
{Definizione di campo elettrico}
{Differenza tra vettori, forma cartesiana}
{Campo elettrico generato da una carica puntiforme}
{Energia cinetica di traslazione}
{Velocità tangenziale}
{Accelerazione centripeta}
{Costante dielettrica del vuoto}
{Prodotto di uno scalare per un vettore, forma cartesiana}
{Costante elettrica del vuoto}
{Modulo di un vettore, note le componenti}
{Quantità di moto}
{Costante di Coulomb}
{Frequenza del moto circolare}
{Forza subita da una carica in un campo elettrico}
{Direzione di un vettore}
{Accelerazione media}
{Accelerazione di gravità sulla superficie della Terra}
{Velocità media}
{Carica elementare}
{Prodotto scalare, forma polare}
{Pressione}
{Primo principio della termodinamica}
{Velocità angolare (pulsazione) del moto circolare}
{Condizione di equilibrio per corpi puntiformi}
{Forza centripeta}
{Leggi del moto rettilineo uniforme}
{Attrito statico}
{Pulsazione dell'onda}
{Riflessione totale, angolo limite}
{Densità di un corpo}
{Legge della rifrazione (legge di Snell)}
{Conversione Celsius-Kelvin}
{Indice di rifrazione di un mezzo materiale}
{Funzioni goniometriche}
{Prodotto vettoriale, forma polare}
\end{enumerate}





BLU
\begin{enumerate}
{Condizione di equilibrio per corpi puntiformi}
{Legge oraria delle onde in un punto fissato}
{Legge della rifrazione (legge di Snell)}
{Direzione di un vettore}
{Campo elettrico generato da una carica puntiforme}
{Accelerazione centripeta}
{Somma tra vettori, forma cartesiana}
{Legge fondamentale della calorimetria}
{Velocità della luce nel vuoto}
{Forza di richiamo di una molla (legge di Hooke)}
{Frequenza}
{Energia cinetica di traslazione}
{Costante elettrica del vuoto}
{Definizione di lavoro}
{Pulsazione dell'onda}
{Potenza media}
{Definizione di campo elettrico}
{Energia potenziale gravitazionale}
{Forza subita da una carica in un campo elettrico}
{Forza centripeta}
{Riflessione totale, angolo limite}
{Velocità media}
{Differenza tra vettori, forma cartesiana}
{Quantità di moto}
{Conversione Celsius-Kelvin}
{Costante dielettrica del vuoto}
{Modulo di un vettore, note le componenti}
{Prodotto scalare, forma polare}
{Primo principio della termodinamica}
{Conversione tra km/h e m/s}
{Il lavoro è una variazione di energia}
{Legge di gravitazione universale}
{Prodotto di uno scalare per un vettore, forma cartesiana}
{Legge di Coulomb}
{Frequenza del moto circolare}
{Forma cartesiana di un vettore}
{Secondo principio della dinamica (legge fondamentale della dinamica)}
{Legge del moto armonico}
{Pressione}
{Velocità tangenziale}
{Densità di un corpo}
{Funzioni goniometriche}
{Indice di rifrazione di un mezzo materiale}
{Carica elementare}
{Forza peso}
\end{enumerate}
\end{document}

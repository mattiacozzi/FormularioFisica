\documentclass[a4paper,11pt,italian]{article}
\usepackage[T1]{fontenc}
\usepackage[utf8]{inputenc}
\usepackage[italian]{babel}
\usepackage{amsmath}
\usepackage{mathrsfs}
\usepackage{hyperref}
%%%--- impostazioni font
\usepackage{sansmathfonts}
\usepackage[scaled=0.85]{helvet}
\renewcommand{\rmdefault}{\sfdefault}
\usepackage[font=footnotesize,width=.75\textwidth]{caption}

%%%--- frazioni carine \sfrac{}{}
\usepackage{xfrac}

%%%--- interlinea
\usepackage{setspace}
\setstretch{1.3}

%%%--- multicolonna
\usepackage{multicol}

%%%--- figure in orizzontale
\usepackage{rotating}

%%%--- margini e bordo
\usepackage[margin=2.7cm]{geometry}
% \usepackage{showframe}

%%%--- impostazioni tikz e grafici
\usepackage{pgf,tikz,pgfplots,bm,pgf-spectra,graphicx,timelines}
\usepackage{circuitikz}
\usetikzlibrary{angles,quotes,arrows,shapes,decorations.markings}
\tikzset{fleche/.style args={#1:#2}{postaction=decorate,decoration={name=markings,mark=at position #1 with {\arrow[#2,scale=1.5]{latex}}},},}
\pgfplotsset{compat=1.15}
\usepgfplotslibrary{units,fillbetween}

%%%--- info documento
\title{\textbf{\Huge \color{colorSitoScuro}Formulario di Fisica}}
\author{\LARGE \color{colorSitoScuro}Mattia Cozzi}
\date{\large \color{colorSitoChiaro}\href{http://mattiacozzi.altervista.org/}{mattiacozzi.altervista.org}}

%%%--- riferimenti intelligenti
\usepackage[noabbrev]{cleveref} %opzione capitalize per tutto maiuscolo

%%%---righelli migliori per tabelle
\usepackage{booktabs}

%%%--- impostazioni dei colori
\usepackage{xcolor}
\usepackage{sectsty}
\usepackage{enumitem}
%%%--- TEAL
\definecolor{colorSito}{rgb}{.04,.212,.22}
\definecolor{colorSitoChiaro}{rgb}{.20,.408,.416}
\definecolor{colorSitoScuro}{rgb}{.0,.133,.141}
%%%--- BORDEAUX
% \definecolor{colorSito}{rgb}{.612,.067,.192}
% \definecolor{colorSitoChiaro}{rgb}{.69,.145,.271}
% \definecolor{colorSitoScuro}{rgb}{.22,.0,.0}

\sectionfont{\color{colorSito}}  %imposta il colore delle sezioni
\subsectionfont{\color{colorSitoChiaro}}  %imposta il colore delle sottosezioni
\subsubsectionfont{\color{colorSitoChiaro}}  %imposta il colore delle sottosottosezioni
\setlist[description]{format=\textcolor{colorSitoScuro}} %imposta il colore degli elementi di una lista di definizione

%%%--- nuovi comandi

\begin{document}
\begin{enumerate}
\item Funzioni goniometriche
\item Scomposizione di un vettore
\item Forma cartesiana di un vettore
\item Modulo di un vettore, note le componenti
\item Direzione di un vettore 
\item Somma tra vettori, forma cartesiana
\item Differenza tra vettori, forma cartesiana
\item Prodotto di uno scalare per un vettore, forma cartesiana
\item Prodotto scalare, forma polare
\item Prodotto vettoriale, forma polare
\item Densità di un corpo
\item Velocità media
\item Conversione tra km/h e m/s
\item Accelerazione media
\item Leggi del moto rettilineo uniforme
\item Leggi del moto accelerato
\item Frequenza del moto circolare
\item Velocità angolare (pulsazione) del moto circolare
\item Velocità tangenziale
\item Accelerazione centripeta
\item Forza centripeta
\item Legge del moto armonico 
\item Secondo principio della dinamica (legge fondamentale della dinamica)
\item Condizione di equilibrio per corpi puntiformi
\item Forza peso
\item Attrito statico
\item Forza di richiamo di una molla (legge di Hooke)
\item Definizione di lavoro
\item Il lavoro è una variazione di energia
\item Potenza media
\item Energia cinetica di traslazione
\item Energia potenziale gravitazionale
\item Quantità di moto
\item Legge di gravitazione universale
\item Accelerazione di gravità sulla superficie della Terra 
\item Pressione 
\item Conversione Celsius-Kelvin
\item Legge fondamentale della calorimetria
\item Primo principio della termodinamica
\end{enumerate}

%39 formule

GIALLI
\begin{enumerate}
{Accelerazione di gravità sulla superficie della Terra}
{Accelerazione media}
{Forma cartesiana di un vettore}
{Energia cinetica di traslazione}
{Velocità media}
{Energia potenziale gravitazionale}
{Modulo di un vettore, note le componenti}
{Velocità angolare (pulsazione) del moto circolare}
{Scomposizione di un vettore}
{Legge di gravitazione universale}
{Prodotto scalare, forma polare}
{Accelerazione centripeta}
{Frequenza del moto circolare}
{Funzioni goniometriche}
{Primo principio della termodinamica}
{Potenza media}
{Secondo principio della dinamica (legge fondamentale della dinamica)}
{Forza centripeta}
{Leggi del moto accelerato}
{Quantità di moto}
{Conversione Celsius-Kelvin}
{Somma tra vettori, forma cartesiana}
{Legge fondamentale della calorimetria}
{Legge del moto armonico}
{Differenza tra vettori, forma cartesiana}
{Attrito statico}
{Leggi del moto rettilineo uniforme}
{Condizione di equilibrio per corpi puntiformi}
{Velocità tangenziale}
{Forza di richiamo di una molla (legge di Hooke)}
{Densità di un corpo}
{Conversione tra km/h e m/s}
{Definizione di lavoro}
{Il lavoro è una variazione di energia}
{Direzione di un vettore}
{Forza peso}
{Prodotto di uno scalare per un vettore, forma cartesiana}
{Pressione}
{Prodotto vettoriale, forma polare}
\end{enumerate}



ROSSI
\begin{enumerate}
{Velocità tangenziale}
{Scomposizione di un vettore}
{Accelerazione centripeta}
{Prodotto scalare, forma polare}
{Energia cinetica di traslazione}
{Forza centripeta}
{Conversione Celsius-Kelvin}
{Energia potenziale gravitazionale}
{Attrito statico}
{Legge fondamentale della calorimetria}
{Pressione}
{Potenza media}
{Somma tra vettori, forma cartesiana}
{Velocità angolare (pulsazione) del moto circolare}
{Differenza tra vettori, forma cartesiana}
{Funzioni goniometriche}
{Leggi del moto rettilineo uniforme}
{Prodotto di uno scalare per un vettore, forma cartesiana}
{Accelerazione media}
{Definizione di lavoro}
{Legge del moto armonico}
{Direzione di un vettore}
{Forma cartesiana di un vettore}
{Forza di richiamo di una molla (legge di Hooke)}
{Frequenza del moto circolare}
{Accelerazione di gravità sulla superficie della Terra}
{Conversione tra km/h e m/s}
{Quantità di moto}
{Primo principio della termodinamica}
{Leggi del moto accelerato}
{Densità di un corpo}
{Secondo principio della dinamica (legge fondamentale della dinamica)}
{Condizione di equilibrio per corpi puntiformi}
{Forza peso}
{Prodotto vettoriale, forma polare}
{Il lavoro è una variazione di energia}
{Legge di gravitazione universale}
{Modulo di un vettore, note le componenti}
{Velocità media}
\end{enumerate}




VERDI
\begin{enumerate}
{Forma cartesiana di un vettore}
{Accelerazione media}
{Definizione di lavoro}
{Modulo di un vettore, note le componenti}
{Velocità tangenziale}
{Energia potenziale gravitazionale}
{Energia cinetica di traslazione}
{Scomposizione di un vettore}
{Velocità media}
{Forza peso}
{Forza di richiamo di una molla (legge di Hooke)}
{Differenza tra vettori, forma cartesiana}
{Frequenza del moto circolare}
{Attrito statico}
{Prodotto vettoriale, forma polare}
{Condizione di equilibrio per corpi puntiformi}
{Legge del moto armonico}
{Leggi del moto rettilineo uniforme}
{Somma tra vettori, forma cartesiana}
{Pressione}
{Conversione Celsius-Kelvin}
{Secondo principio della dinamica (legge fondamentale della dinamica)}
{Funzioni goniometriche}
{Potenza media}
{Legge fondamentale della calorimetria}
{Forza centripeta}
{Direzione di un vettore}
{Conversione tra km/h e m/s}
{Prodotto di uno scalare per un vettore, forma cartesiana}
{Il lavoro è una variazione di energia}
{Leggi del moto accelerato}
{Densità di un corpo}
{Velocità angolare (pulsazione) del moto circolare}
{Legge di gravitazione universale}
{Prodotto scalare, forma polare}
{Accelerazione di gravità sulla superficie della Terra}
{Primo principio della termodinamica}
{Quantità di moto}
{Accelerazione centripeta}
\end{enumerate}





BLU
\begin{enumerate}
{Forza peso}
{Velocità tangenziale}
{Frequenza del moto circolare}
{Accelerazione di gravità sulla superficie della Terra}
{Il lavoro è una variazione di energia}
{Legge di gravitazione universale}
{Pressione}
{Prodotto di uno scalare per un vettore, forma cartesiana}
{Condizione di equilibrio per corpi puntiformi}
{Conversione tra km/h e m/s}
{Velocità angolare (pulsazione) del moto circolare}
{Legge fondamentale della calorimetria}
{Direzione di un vettore}
{Modulo di un vettore, note le componenti}
{Energia cinetica di traslazione}
{Primo principio della termodinamica}
{Accelerazione media}
{Differenza tra vettori, forma cartesiana}
{Funzioni goniometriche}
{Leggi del moto accelerato}
{Potenza media}
{Forza centripeta}
{Attrito statico}
{Prodotto vettoriale, forma polare}
{Definizione di lavoro}
{Scomposizione di un vettore}
{Forma cartesiana di un vettore}
{Leggi del moto rettilineo uniforme}
{Accelerazione centripeta}
{Secondo principio della dinamica (legge fondamentale della dinamica)}
{Somma tra vettori, forma cartesiana}
{Prodotto scalare, forma polare}
{Legge del moto armonico}
{Densità di un corpo}
{Energia potenziale gravitazionale}
{Forza di richiamo di una molla (legge di Hooke)}
{Velocità media}
{Conversione Celsius-Kelvin}
{Quantità di moto}
\end{enumerate}
\end{document}

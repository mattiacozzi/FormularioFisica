%%%--- impostazioni font
\usepackage{sansmathfonts}
\usepackage[scaled=0.85]{helvet}
\renewcommand{\rmdefault}{\sfdefault}
\usepackage[font=footnotesize,width=.75\textwidth]{caption}

%%%--- frazioni carine \sfrac{}{}
\usepackage{xfrac}

%%%--- interlinea
\usepackage{setspace}
\setstretch{1.3}

%%%--- multicolonna
\usepackage{multicol}

%%%--- figure in orizzontale
\usepackage{rotating}

%%%--- margini e bordo
\usepackage[margin=2.7cm]{geometry}
% \usepackage{showframe}

%%%--- impostazioni tikz e grafici
\usepackage{pgf,tikz,pgfplots,bm,pgf-spectra,graphicx,timelines}
\usepackage{circuitikz}
\usetikzlibrary{angles,quotes,arrows,shapes,decorations.markings}
\tikzset{fleche/.style args={#1:#2}{postaction=decorate,decoration={name=markings,mark=at position #1 with {\arrow[#2,scale=1.5]{latex}}},},}
\pgfplotsset{compat=1.15}
\usepgfplotslibrary{units,fillbetween}

%%%--- info documento
\title{\textbf{\Huge \color{colorSitoScuro}Formulario di Fisica}}
\author{\emph{\LARGE \color{colorSitoScuro}Mattia Cozzi}}
\date{\large \color{colorSitoChiaro}\href{http://mattiacozzi.altervista.org/}{mattiacozzi.altervista.org}}

%%%--- riferimenti intelligenti
\usepackage[noabbrev]{cleveref} %opzione capitalize per tutto maiuscolo

%%%---righelli migliori per tabelle
\usepackage{booktabs}

%%%--- impostazioni dei colori
\usepackage{xcolor}
\usepackage{sectsty}
\usepackage{enumitem}
%%%--- TEAL (completo)
\definecolor{colorSito}{rgb}{.04,.212,.22}
\definecolor{colorSitoChiaro}{rgb}{.20,.408,.416}
\definecolor{colorSitoScuro}{rgb}{.0,.133,.141}
%%%--- BORDEAUX (semplificato)
% \definecolor{colorSito}{rgb}{.612,.067,.192}
% \definecolor{colorSitoChiaro}{rgb}{.69,.145,.271}
% \definecolor{colorSitoScuro}{rgb}{.22,.0,.0}

\sectionfont{\color{colorSito}}  %imposta il colore delle sezioni
\subsectionfont{\color{colorSitoChiaro}}  %imposta il colore delle sottosezioni
\subsubsectionfont{\color{colorSitoChiaro}}  %imposta il colore delle sottosottosezioni
\setlist[description]{format=\textcolor{colorSitoScuro}} %imposta il colore degli elementi di una lista di definizione

%%%--- differenti sezioni scientifico/nonscientifico
\usepackage{etoolbox}   % for booleans and much more
\usepackage{verbatim}   % for the comment environment
% new environment
\newenvironment{soloscientifico}{}{} %creazione del nuovo ambiente
% set conditional behaviour of environment
\ifbool{semplificato}{\AtBeginEnvironment{soloscientifico}{\comment}% 
\AtEndEnvironment{soloscientifico}{\endcomment}}{}

%%%--- nuovi comandi
